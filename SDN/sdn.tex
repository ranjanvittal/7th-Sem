\documentclass[solution,addpoints,12pt]{exam}
\printanswers
\usepackage{amsmath,amssymb}
\usepackage[T1]{fontenc}
\begin{document}
Coursera : Nick Feamster.\\
Network protocol layers :\\
Application layer, Transport layer : end to end and reliable,
Network layer : routing , congestion control, addressing and Quality of service
Link layer : link by link basis, Physical layer.\\

Components of a router :\\
\begin{itemize}
\item Line cards.\\
\begin{itemize}
\item
Each line card is for one particular file layer technology.
Based in transmitter/Recv).
\item
Line cards these days have CPU and a memory dedicated to it.
\item
It also takes care of forwarding. A packet based on the input
header and route it to the next router on the route to the
destination router. From the input port which output port it goes out
in this router. Based on forwarding table it does all these.
A simple forwarding table will have an input port mapped to the output port
based on the header information. Address has two parts : network address
and host address. Number of possible networks is the bound for the number
of entries in the forwarding table.
\end{itemize}
\item Control Card : Also has CPU and memory. Handles all control packets.
Computes and loads the forwarding table.
\item Management Card : Also has CPU and memory. Handles all monitoring
packets.
\item Switch fabric connects the different cards.
\end{itemize}

There are some packets which could be control packet like non data
packet which is for routing protocols like OSPF. These are called control
packets. RIP, REP and ISIS are other routing protocols. An autonomous system
is the set of systems under one name (like reliance's systems under
reliance name). BGP is the border gateway protocol is like sendin packt
from one cluster to the next cluster. RIP is the protocol for internal
transfer.

We have various configurations possible which can be configured using SMP.\\
Some are simply forwarded like data packets. The other are used to make
some changes to the memory of the router.

Data plane or the forwarding plane, control plane and forwarding plane.\\
Routers do look at TCP header.\\

Innovations inside the network were growing at a slow rate.\\
Physical layers has been growing over the years. Wifi has
grown from 2 Mbps to about 1 Gbps. Cellular receives
about 5Mbps. Like from 2g to 3g to 4g the speeds increase.
The base state for optic networks was 155 Mbps. It has gone
upto 100 Gbps per channel. Hence it has gone to about 10 Tbps
on a link.\\

Application also has proceeded nicely. Changing things easily
is possible.\\
FIND : Future Intermediate Directions.\\
NGI : Next Generation of Internet.\\
Clean State Internet.\\
Goals :\\
\begin{itemize}
\item Enable faster innovation in a production/live network.
People who want the above are vendors or manufactureres of network
equipment like cisco, juniper, tejas, huawei. The others who want
these are Internet Service Providers. They take the equipment
from the vendors and give internet service. The ISP has to give
some service agreement. If they want to add a new service they
have to go to the equipment vendor and ask for it. Based on something
already existing, if an innovation is happening, then it must be done quickly.
\item Network element programmability : An administrator should
be able to write high level programs to modify the network element
behaviour.
\item Given a large scale network, formally verifying connectivity between two
other nodes and such.\\
\item How to eliminate proliferation of middle boxes.\\
Middle box is a hardware sitting in the network between routers,
basically anywhere between soure and destination.\\
They enable :
\begin{itemize}
\item firewall.
\item proxies / caching.
\item NAT(Network Address Translation) / Gateway.
\item Load balancing.
\item Intrusion detection or prevention.
\item WAN optimizers/Application accelerators.
\item This might have the power to send acks, etc to the nodes in order
to change the parameters like recevier window size of the nodes.
\end{itemize}
\end{itemize}
The players involved are :\\
Internet service providers.\\
Enterprise nets are private networks like IIT, University of texas, etc.\\
Data center networks basically have racks of servers. They have almost
2 million IP addresses.\\
Content delivery network.\\

Network virtualization :\\
Jon turner made effort to incorporate network virtualization.
Host OS over hardware was the initial stage. VMWare
added a virtual box over Host OS. Now we can have linux running
over windows and such. For cloud computing VMWare is necessary.
(Using different versions over the underlying model of old machines)\\
We can start assigning cores to different versions. Slices
of the computer infrastructure is the one discussed above.\\

The slices of network Infrastructure is network virtualization.
Each slice can run its own protocol stack. PlanetLAB was the
first testbed which can be used for wide area testing.
Now there are GENI testbeds.\\

Network function Virtualization (NFV) will be discussed later.\\

Read about : Active networking, FORCES(IETF), Routing Control Protocol(RCP),
Ethane project in stanford and write a one page report on each of them.\\

Notions of SDN :\\
Separation of control plane and data plane :
Router's internals include :\\
\begin{itemize}
\item Communications hardware
\item CPU, Memory, Buffers, etc
\item Switching fabric
\item Router operating systems : Cisco's (IOS), JunOS.
Open Networking Linux is another OS.
\item Features :
\begin{itemize}
\item OSPF
\item RIP
\item daemon
\item BGP
\end{itemize}
\end{itemize}
The above are Vetically Integrated Closed Proprietory System.\\
The control plane is tightly interconnected with the data
plane in the above.\\

Example of network virtualization:\\
An user wants a different set of protocols for the system.
The routers have hardwares for these. Now a user must be
able to run the protocols which they want to run on.
Different operators working on the same underlying interface.

SDN wants better. The integrated part was split into two
parts as control and data plane.\\
The idea was to use the switch/router only implements the data plane
(ie) the packet forwarding aspects only.\\

Forward based on :\\
\begin{itemize}
\item
Destination address only are
split into class based IP addressing(Fixed length prefix
and has an exact matching to the incoming address)
and the other is the classless IP addressing(longest matching
prefix). Class based is more like a cluster while classless
is not so.
\item
Based on several fields in the IP. This is actually
breaking the IP stack rule. Based on
<Src IP, Dest IP, Src Port, Dest Port, Protocol>. This has 5 fields.
Therefore the routing becomes extremely complicated due to all these.
\item
The following is called the policy based routing which has about 40 fields :\\
Forwarding data plane has a forwarding table. There may be various
fields in the packet header. The packet has MAC, IP, TCP headers
which in themselves have several fields ending up in about 40 fields.
Based on the fields, we have action and auxillary actions. Because of these
the forwarding table might become exponential. Hence we have some kind
of aggregation for every field. There will always be a default rule.
This exists now. A given packet can match multiple entries in the forwarding
table.\\
100 Gbps. TCP : 40 byte minimum packet size (TCP + IP). 40 bytes and 1500
bytes are where the spikes of statistics of data sent.\\
Time available to process a packet in the
forwarding table = (40*8)/(100 Gbps) = 3.2 ns. Therefore the processing time
should be less than this for the packets not to queue up.\\

The same packet might get different treatment at different routers.
\end{itemize}

Telephone networks :\\
The paths have been done via circuit switching. The controller
decides these paths. The controller has the global view. This one is
centralized routing.\\

OSPF is distributed routing as there is
no one centralized node. Hence the distributed one is where each
router has the ability to route on its own without the
help of the centralized node.\\

Current challenge is to have some controllers but not one like
centralized controller.\\

SDN model : Switch or router only implements the data plane functionality.\\
The control plane line card is to be removed from the router.\\
A well defined interface to the control plane. The controller
is now being implemented as a software. Therefore this is moving towards
programmability. The interface between control and data plane must be
open and standardized. Only if that happens the communication between
different companies would be easier. The controller is central for
some routers and it is not specific to a router as before. Example for
open interface is open flow. This is called south bound interface.\\

East west is between servers traffic and  North south is between the server
and the internet in data center terms.\\

The applications that talk to the controller software is called the
north bound interface. Different south bound interfaces are NOX,
POX, Beacon, Floodlight, OpenDayLight, ONOS.\\

The switch and router are to be designed to be dumb devices.
The only functionalities are looking up forwarding table.
Implementing packet classification in hardware :\\
First do a good algorithm in software and translate part of
it as hardware. Do an ASIC/FPGA implementation of an algorithm.\\
FPGA : Field Programmable Gate Array.\\
ASIC : Application Specific Integrated Circuit.\\
FPGA is slightly better. We can modify easily.\\
Hardware lookup tables are also better.
These are called ternary content address memory.\\
It is called ternary because 0, 1 or dont care.\\
TCAM is also programmable. TCAM is very fast, but costly
and consumes more power. We must also try to ensure
lesser size in TCAMs. Now the data plane has this
hardware dedicated for it. This is a very simple router
and hence it would become a commodity and the USP of
CISCO, etc breaks. The vendors dont want to make dumb
routers.\\

Active networking : Every packet has embedded code
in it which tells the router how to progress. Specialized
instruction could be given to the router for it to process
the router. Read about IETF.\\

FORCES : separation of control and data within or outside of a router.\\
Ethane giving rise to OpenFlow.\\
Look at the slides. (History of programming networks).
B - Broadcast, U - Unicast,
M - Multicast which is specialized way of combining
local networks. Like a match being streamed to one machine
and several people near the machine gets the copies. So long
range transmissions are enabled better.
MBone was designed for multicast over the underlying network.
They had specialized routers which did muliticast routing. They
were called application level routers. This was considered because
the underlying stuff need not be changed.\\
6Bone built IPV6 over IPV4. IPV6 is fixed length 40 byte header.
IPV4 is variable length header. Similar to MBone
they built IPV6 application level routers. Most of
these were application level overlay.\\

Application to SDN controller(Control plane) to routers(Data plane).
Standards are made by ONF. The application layer needs a consistent
interface to the controller. The management and administration
planes has interfaces to all the three. Management supplies software
updates to the routers, it configures policy and monitors performance
with the control layer.\\

Applications of SDN:\\
Data center networks, Enterprise networks, Wireless Access Networks,
Optical networks, home and small business networks.\\

(Look at picture in slide)\\
Green arrows : When the entry is present router goes through these.\\
Blue arrows : When it doesnt know how to handle.\\
Scalability in terms of number of new flows is a huge bottle neck.\\
Data center networks :\\
Elephant flow : The number of flows is small and the bandwidth is huge.
Backups, distributed computation (east west traffic). Most of
the traffic is inside the data center. VM migration
also takes a lot of time.\\
Mice flows : There are lot of such flows but very small bandwidths.\\

OpenFlow protocol (Read from slides).\\
A set of rules define flows like a tuple matching instead
of just src IP and destination IP.\\
\begin{itemize}
\item Switch has two parts, Datapath and the SDN client.
\item The communication between the client and the controller
is generally TLS over TCP.
\item TLS is transport layer security. We might TCP alone if we want
faster performance. Forwarding element implements CPU like instructions.
\item These were for Ethernet switches initially. Now they have
extended to other switches and routers. Current version is 1.4.
Openflow version 1.0 came in 2009.
\end{itemize}

Openflow Specs :\\
\begin{itemize}
\item In version 1.0 there were 12 fields.
\item It had only one flow table. As we went on more
flow tables were added for easier access or specialized
processing.
\item Initially the table is empty we learn the mac address
of every port. If a new mac address comes in the controller
is notified and a broadcast of the mac is made.
\item Src Port, Destination Port, IP Protocol, Type of Service
and 8 more fields are there in the flow table.
\item There are many different actions which could be there
for processing of a packet. Statistics are also collected based on what
is done and would be given to the management plane.
Actions are forward or drop in the first version.
Special actions are Enqueue, modify. A flow table can
specify multiply actions for a given packet. A port
can have multiple queues. Modify packet headers can also be
done.
\item Packets are of two types:\\
From a host or from the controller. If it is from the controller,
we might need to update the router.
\item If there are more than one flow table, the packet
will go through all the flow tables and collect all the actions
and do it finally.
\item We might need to modify the src IP even while forwarding
listed as an action (ie) modify packet. Update action
set, update metadata are also different instructions to be
applied. Controller is a piece of software.
\item Now there are hardware switches implementing OpenFlow.
\item Interoperability is majorly maintained.
\item There are special virtual ports like LOCAL, ALL, CONTROLLER
specifying where it must be sent. IN_PORT meaning send packet
back on input port. TABLE packets sent by controller
forwarded either on specified output in PACKET_OUT message or the
Flow Table.FLOOD : Similar to ALL except that it send on links of
the STP tree. NORMAL is the normal processing.
\end{itemize}

Hubs simply forward whatever it gets to all the nodes.
Group table is a set of actions specified.
OXM Header followed by TLV packets. The OXM header contains
4 bytes which also has the length of the overall packet which is
from 5 to 259 bytes (1 to 255 otherwise).\\
There are master and slave controllers. When master fails,
the slave takes over. There is a concept of equal controller
where two controllers can access the same router.\\
Data path id is the connection number of the router to the
controller.\\
Wildcard matching can be used to bypass some tables.\\
There are ways to look at the meter reading of how many bytes
are received and such statistics.\\
Firefox : downthemall option for faster download due to parallelization.\\
Similarly there are multiple connections between the controller and the
switch.\\
Controller caches data of PACKET\_IN and uses it later. There
is a cookie along with that packet.\\

Further versions of Open Flow also had mechanisms to
notify the controller and ask which rules to evict.
An atomic bundle mechanism was also developed where
we either execute all the actions within or none of them.\\

Networks supporting tenants.\\
One tenant using 10.* and another also wants to use 10.* .
How to implement something like this.\\

Look at all possible header space. In flow visor,
you can split your header space into multiple slices.
We take some slices and the remaining is considered the master slice.\\
\begin{itemize}
\item Policy language to map flows to slices.
\item Resource can be sliced in terms of BW, Topology, Device CPU,
Flow table entries and so on.
\item Can operate on deplayed on networks. The production
network can be a testbed.
\item Slices work on different types of services.
\item We can have multiple controllers and have each controller correspond to
one slice.
\end{itemize}
The controller thinks the flow visor is the router and router thinks
of the flow visor as the controller. The individual controllers
are aware that they are part of the slice network though.\\

Flow visor controls the controllers and ensure that they are mutually
exclusive. A new controller can be added while the other
controllers are notified. There might be multiple controllers
implementing the same slice which the flow visor will manage.\\
You can specify a virtual topology which would be embedded on a physical
topology. This can be different. The mapping would end up being
complicated. We can also specify the parameters for the topology.\\

Virtual links can be made up of multiple physical links.
There is a possibility of same address space being shared across
different virtual topologies. Different topologies may be
implemented/controlled by different controllers. The OVX
is the interface between the controllers and the switches.\\

For mice flows we require less latency and bandwidth can be
compromised, while it is the reverse for elephant flows.
Fat tree means more connections to the higher level switches.
The hierarchy is Edge(or Top of the rack) switch to aggregation switch to core
switch.\\

Over subscription could be an issue if all children keep producing to their
best which might end up creating a problem. These are between the
racks in the servers.
\end{document}
